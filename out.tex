%% Setup to use an index card
\documentclass[10pt]{book}
\usepackage[vcentering,dvips]{geometry}
\geometry{papersize={3in,5in},total={2.9in,4.9in}}

%% render a frame marking the margins of a document
% \usepackage{showframe}

%% show landscape view
\usepackage{pdflscape}

%% Use for testing by filling junk (lorem-ipsum) information
\usepackage{lipsum}

\begin{document}
\begin{landscape}

%% Use for testing by filling junk (lorem-ipsum) information
\textbf{ID: 1.1} \emph{Context: P3 Speaker:R}\\ 
R: So let's go back. When you say visual research what do you mean by visual research? \newpage 

\textbf{ID: 2.1} \emph{Context: P4 Speaker:A}\\ 
A: Um. I'm going to put this in really simple terms and then you are going to say, thanks for that! So as a lighting designer, lighting is really really difficult for people to talk about you know because we are inundated with it. \newpage 

\textbf{ID: 2.2} \emph{Context: P4 Speaker:A}\\ 
and so I could say you're going to have a 30 degree barrel and we're going to put rosco 02 out there and it's going to be 45 degrees up and out most directors also go I don't know. Most directors also go I don't have a clue what you're talking about. \newpage 

\textbf{ID: 2.3} \emph{Context: P4 Speaker:A}\\ 
Um and so as a designer that's not a particularly useful way to communicate and the director but if I can show them a picture that's close, something that even just gets the mood across. That's how I communicate as a lighting designer. And and mostly it's about mood rather than specific angles, mood and color. \newpage 

\textbf{ID: 2.4} \emph{Context: P4 Speaker:A}\\ 
Sometimes you get really great angles but in the theater we are a bit limited as to where the light coming from the amount of bounce and things that you have. And most designers are doing visual research in this way. Mood imagery that may or may not have anything to do with the play itself except, you know, it's a dark and stormy night. \newpage 

\textbf{ID: 3.1} \emph{Context: P5 Speaker:R}\\ 
R: It sound's really hard to me [A laughing] to do that \newpage 

\textbf{ID: 4.1} \emph{Context: P6 Speaker:R}\\ 
R: What does your meeting look like. It sounded like maybe part of the reason you have to communicate in this medium is so that people who aren't as familiar with lighting design can know what you're talking about. \newpage 

\textbf{ID: 5.1} \emph{Context: P7 Speaker:A}\\ 
A: Typically in a production, we have production meetings with \newpage 

\textbf{ID: 6.1} \emph{Context: P8 Speaker:A}\\ 
A: And, so, we are all very much specialists which means going into that meeting you do have to sort of communicate where you are going to go without having to be super specific, because no one else in that room is a lighting designer. \newpage 

\textbf{ID: 7.1} \emph{Context: P9 Speaker:R}\\ 
R: I imagine there's a sense of trust there, between you guys. Okay, I understand this direction I think I know what you're talking about, I like this picture, and that's good enough for me right now. \newpage 

\textbf{ID: 8.1} \emph{Context: P10 Speaker:A}\\ 
A: Yeah, I mean, outside of the university setting. Someone has looked through my portfolio \newpage 

\textbf{ID: 9.1} \emph{Context: P11 Speaker:and will know the quality of that work. They aren't going to know necessarily that I showed that picture [points to picture] and got that image on stage. But, You know you look at the overall end product in quality and know you clearly know what you are doing and have lit. In the outside world that's how you get hired. In education, you sort of work with what you get and hopefully these are all you know good colleges that are are working at the same time.}\\ 
and will know the quality of that work. They aren't going to know necessarily that I showed that picture [points to picture] and got that image on stage. But, You know you look at the overall end product in quality and know you clearly know what you are doing and have lit. In the outside world that's how you get hired. \newpage 

\textbf{ID: 9.2} \emph{Context: P11 Speaker:and will know the quality of that work. They aren't going to know necessarily that I showed that picture [points to picture] and got that image on stage. But, You know you look at the overall end product in quality and know you clearly know what you are doing and have lit. In the outside world that's how you get hired. In education, you sort of work with what you get and hopefully these are all you know good colleges that are are working at the same time.}\\ 
In education, you sort of work with what you get and hopefully these are all you know good colleges that are are working at the same time. \newpage 

\textbf{ID: 10.1} \emph{Context: P12 Speaker:A}\\ 
A: And over time, because we tend to have the same design team, you know you you develop a shorthand. And so, you know, there's no question to me, working with certain directors that if I show them an image that may not have great specific light detail in it, that I can make that mood happen on stage. So there is a lot of trust there.[laughs] \newpage 

\textbf{ID: 11.1} \emph{Context: P13 Speaker:R}\\ 
R: So as you're using Pinterest in what context... In what context might you use it, locally? Sorry, let's get back to the meeting. What is that space? \newpage 

\textbf{ID: 12.1} \emph{Context: P14 Speaker:A}\\ 
A: Traditionally we are in a classroom space that has a projector. [12:45] Um, so we have access to computers and a projection screen. We as a department are trying to go much greener in general, and not have so much waste.  It has been both beneficial and troubling at times. You see I have all of these binders. and those are all different productions. \newpage 

\textbf{ID: 12.2} \emph{Context: P14 Speaker:A}\\ 
Most of them have a big chunk of research in them just visuals that I had to design and present in a production meeting. I have to find a way to ultimately save that information if I'm ever going to go back to that show. Now I just have DVDs and save those images not on Pinterest but, Pinterest kind of falls apart for me, because there's not an easy way to save that information. \newpage 

\textbf{ID: 13.1} \emph{Context: P15 Speaker:R}\\ 
R: As someone who tries to design things that are easy that made me cry a little on the inside. [A laughs, says I understand.] Right click, save. \newpage 

\textbf{ID: 14.1} \emph{Context: P16 Speaker:R}\\ 
R: So those differences are really interesting I can see you have 20 or so binders, you have the processes logged. We talked about the painful process of saving the DVDs. Have you ever gone back and look at one of the images on the DVD? \newpage 

\textbf{ID: 15.1} \emph{Context: P17 Speaker:A}\\ 
A: I have, mostly for teaching purposes. I have yet to redo a show. But sometimes, when looking for inspiration for a show, and I know that have this picture somewhere that I put it in a binder somewhere. So more often than not, I'm using them to talk through my process to my classes. Also, all of the information saved on DVDs is also saved on my computer somewhere. Like I haven't just deleted it. \newpage 

\textbf{ID: 15.2} \emph{Context: P17 Speaker:A}\\ 
I have gone through that information. I know I have a really great picture of this thing somewhere. \newpage 

\textbf{ID: 16.1} \emph{Context: P18 Speaker:R}\\ 
R: What about your books? How would you.... how do you utilize the tangible artifacts and in what context would you? When you are developing ideas? or ......[16:00] I know I have this idea in these books somewhere. \newpage 

\textbf{ID: 17.1} \emph{Context: P19 Speaker:A}\\ 
A: That's typically where it is. You know I have a really great picture of a downlight. And it's And typically I know which show it is associated with, unless I have moved it to another show. I have a sense of what this image is and then it ties it back to a show from there. And so. One of the ways we teach our students is to talk about creating morgues. Um, morgues of images. \newpage 

\textbf{ID: 17.2} \emph{Context: P19 Speaker:A}\\ 
Yeah I know it sounds a little creepy. Um, but, to sort of get that recall but also know that you have a set of images, you have a set of ideas, that you know have already analyzed these pictures. And that's' essentially what all of those books are in the end. I could probably take all of the research out and put it into one place, because there's a lot of other stuff in those binders as well. \newpage 

\textbf{ID: 17.3} \emph{Context: P19 Speaker:A}\\ 
But to create a central sort of research place is really useful because you know you already done the initial work of analyzing that image so you don't have to figure out where that light is coming from. \newpage 

\textbf{ID: 18.1} \emph{Context: P20 Speaker:R}\\ 
R: I see, bring it back to life \newpage 

\textbf{ID: 19.1} \emph{Context: P21 Speaker:R}\\ 
R: So why morgue? \newpage 

\textbf{ID: 20.1} \emph{Context: P22 Speaker:A}\\ 
A: It's just a collection of. \newpage 

\textbf{ID: 21.1} \emph{Context: P23 Speaker:R}\\ 
R: A collection of corpuses? It's a pun? \newpage 

\textbf{ID: 22.1} \emph{Context: P24 Speaker:A}\\ 
A: [laughing] Yes it is. The students are always like Uhh... \newpage 

\textbf{ID: 23.1} \emph{Context: P25 Speaker:R}\\ 
R: We were talking about um, this meeting and we have projector on a screen. What else happens? You have a projector on the screen and maybe bring in Pinterest at what point? \newpage 

\textbf{ID: 24.1} \emph{Context: P26 Speaker:A}\\ 
A: Often we start by talking through what your individual design concept is? So the director will always have some kind of overall production concept. Then all of the designers within that have their own ideas. \newpage 

\textbf{ID: 24.2} \emph{Context: P26 Speaker:A}\\ 
So the lighting is going to fall from warm to cool as the relationship falls apart and we are going to kind of shift angles so that we get a lot more shadows on the faces so it's not as clean and bright as it was when it started. \newpage 

\textbf{ID: 25.1} \emph{Context: P27 Speaker:A}\\ 
A: So share you design concept as a whole, and pull up Pinterest and say, in the beginning we don't have many shadows everything is warm and sunny. Here's an image of a beautiful happy day. Here is the color we are going to put on the actors skin. \newpage 

\textbf{ID: 25.2} \emph{Context: P27 Speaker:A}\\ 
This is the overall mood we are trying to get to, which would be one pinboard, probably, and then shift to the end of the play and go to that pinboard or go through the individual element. If there are particular elements, specifically with lighting, I typically can't find one image that has the right color with the right angle with exactly the right mood. \newpage 

\textbf{ID: 25.3} \emph{Context: P27 Speaker:A}\\ 
And so I'm going to have to go through several images to communicate all of that information. \newpage 

\textbf{ID: 26.1} \emph{Context: P28 Speaker:A}\\ 
A: But what's great about Pinterest is that when you do look at that board as a whole, instead of the individual pictures, you get an overall sense of where you're going. Um. Whereas with individual pictures what I had to used to have to do is basically print them out and collage them on the table so you get that same sense. \newpage 

\textbf{ID: 26.2} \emph{Context: P28 Speaker:A}\\ 
What I do find frustrating about Pinterest, and maybe I just haven't played with it enough, because lord knows I know a lot of people that will spend hours and hours on Pinterest  And I'm really using it as a tool to communicate. But find it frustrating that I cannot collage them. to share that information in the way that I want. \newpage 

\textbf{ID: 26.3} \emph{Context: P28 Speaker:A}\\ 
And so it's sort of whatever got there first and then I have to flip through. So you still get a sense of college but not the way I would have designed it. Does that make sense? \newpage 

\textbf{ID: 27.1} \emph{Context: P29 Speaker:R}\\ 
R: That makes total sense. I don't think you are even able to change the ordering. It is chronological. \newpage 

\end{landscape}
\end{document}